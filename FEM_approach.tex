\section{A finite elements method approach}

\subsection{A brief overview}

Assume a general partial differential equation (PDE)
% MathType!MTEF!2!1!+-
% feaagKart1ev2aaatCvAUfKttLearuqr1ngBPrgarmqr1ngBPrgitL
% xBI9gBamXvP5wqSXMqHnxAJn0BKvguHDwzZbqegm0B1jxALjhiov2D
% aeHbuLwBLnhiov2DGi1BTfMBaebbfv3ySLgzGueE0jxyaibaiKc9yr
% Vq0xXdbba91rFfpec8Eeeu0xXdbba9frFj0-OqFfea0dXdd9vqaq-J
% frVkFHe9pgea0dXdar-Jb9hs0dXdbPYxe9vr0-vr0-vqpWqaaeaabi
% GaciaacaqabeaadaabauaaaOqaaiabdYeamjabdAgaMjabgkHiTiab
% dEgaNjabg2da9iabicdaWaaa!47C0!
\begin{equation}
Lf - g = 0
\end{equation}
with boundary conditions
% MathType!MTEF!2!1!+-
% feaagKart1ev2aaatCvAUfKttLearuqr1ngBPrgarmqr1ngBPrgitL
% xBI9gBamXvP5wqSXMqHnxAJn0BKvguHDwzZbqegm0B1jxALjhiov2D
% aeHbuLwBLnhiov2DGi1BTfMBaebbfv3ySLgzGueE0jxyaibaiKc9yr
% Vq0xXdbba91rFfpec8Eeeu0xXdbba9frFj0-OqFfea0dXdd9vqaq-J
% frVkFHe9pgea0dXdar-Jb9hs0dXdbPYxe9vr0-vr0-vqpWqaaeaabi
% GaciaacaqabeaadaabauaaaOqaaiabdoeadjabdAgaMjabgkHiTiab
% dIgaOjabg2da9iabicdaWaaa!47B0!
\begin{equation}
Cf - h = 0
\end{equation}
The Galerkin finite elements method (FEM) solves this problem
in a weak form by integrating by parts the following equation:
% MathType!MTEF!2!1!+-
% feaagKart1ev2aaatCvAUfKttLearuqr1ngBPrgarmqr1ngBPrgitL
% xBI9gBamXvP5wqSXMqHnxAJn0BKvguHDwzZbqegm0B1jxALjhiov2D
% aeHbuLwBLnhiov2DGi1BTfMBaebbfv3ySLgzGueE0jxyaibaiKc9yr
% Vq0xXdbba91rFfpec8Eeeu0xXdbba9frFj0-OqFfea0dXdd9vqaq-J
% frVkFHe9pgea0dXdar-Jb9hs0dXdbPYxe9vr0-vr0-vqpWqaaeaabi
% GaciaacaqabeaadaabauaaaOqaamaapebabaWaaeWaaeaacqWGmbat
% daaeabqaaiabew9aMnaaBaaaleaacqWGPbqAaeqaaOGaemOray0aaS
% baaSqaaiabdMgaPbqabaGccqGHsislcqWGNbWzaSqabeqaniabggHi
% LdaakiaawIcacaGLPaaacqaHvpGzdaWgaaWcbaGaem4AaSgabeaaki
% abbsgaKjabfM6axbWcbaGaeuyQdCfabeqdcqGHRiI8aOGaeyypa0Ja
% eGimaaJaeiilaWIaaGzbVlabdUgaRjabg2da9iabigdaXiablAcilj
% abd6eaobaa!61F7!
\begin{equation}
\int_\Omega  {\left( {L\sum {{\phi _i}{F_i} - g} } \right){\phi _k}{\text{d}}\Omega }  = 0,\quad k = 1 \ldots N
\end{equation}
Here, $\phi_i$ are the finite elements, which are basis functions with
finite support, and the function $f$ is discretized as
% MathType!MTEF!2!1!+-
% feaagKart1ev2aaatCvAUfKttLearuqr1ngBPrgarmqr1ngBPrgitL
% xBI9gBamXvP5wqSXMqHnxAJn0BKvguHDwzZbqegm0B1jxALjhiov2D
% aeHbuLwBLnhiov2DGi1BTfMBaebbfv3ySLgzGueE0jxyaibaiKc9yr
% Vq0xXdbba91rFfpec8Eeeu0xXdbba9frFj0-OqFfea0dXdd9vqaq-J
% frVkFHe9pgea0dXdar-Jb9hs0dXdbPYxe9vr0-vr0-vqpWqaaeaabi
% GaciaacaqabeaadaabauaaaOqaaiabdAgaMjabg2da9maaqaeabaGa
% emOray0aaSbaaSqaaiabdMgaPbqabaGccqaHvpGzdaWgaaWcbaGaem
% yAaKgabeaaaeqabeqdcqGHris5aaaa!4B6B!
\begin{equation}
f = \sum {{F_i}{\phi _i}} 
\end{equation}

The integration by parts has an important property: it can
remove derivatives in $L$, i.e. in the coefficients that appear
in the original equations, as
% MathType!MTEF!2!1!+-
% feaagKart1ev2aaatCvAUfKttLearuqr1ngBPrgarmqr1ngBPrgitL
% xBI9gBamXvP5wqSXMqHnxAJn0BKvguHDwzZbqegm0B1jxALjhiov2D
% aeHbuLwBLnhiov2DGi1BTfMBaebbfv3ySLgzGueE0jxyaibaiKc9yr
% Vq0xXdbba91rFfpec8Eeeu0xXdbba9frFj0-OqFfea0dXdd9vqaq-J
% frVkFHe9pgea0dXdar-Jb9hs0dXdbPYxe9vr0-vr0-vqpWqaaeaabi
% GaciaacaqabeaadaabauaaaOqaamaapeaabaWaaeWaaeaadaWcaaqa
% aiabgkGi2cqaaiabgkGi2kabdIha4baacqWGebardaWcaaqaaiabgk
% Gi2kabdAgaMbqaaiabgkGi2kabdIha4baaaiaawIcacaGLPaaacqaH
% vpGzcqqGKbazcqWG4baEaSqabeqaniabgUIiYdGccqGH9aqpcqWGeb
% ardaWcaaqaaiabgkGi2kabdAgaMbqaaiabgkGi2kabdIha4baacqaH
% vpGzcqGHsisldaWdbaqaaiabdseaenaalaaabaGaeyOaIyRaemOzay
% gabaGaeyOaIyRaemiEaGhaamaalaaabaGaeeizaqMaeqy1dygabaGa
% eeizaqMaemiEaGhaaiabbsgaKjabdIha4bWcbeqab0Gaey4kIipaaa
% a!7023!
\begin{equation}
\int {\left( {\frac{\partial }{{\partial x}}D\frac{{\partial f}}{{\partial x}}} \right)\phi {\text{d}}x}  = D\frac{{\partial f}}{{\partial x}}\phi  - \int {D\frac{{\partial f}}{{\partial x}}\frac{{{\text{d}}\phi }}{{{\text{d}}x}}{\text{d}}x} 
\end{equation}
These derivatives are often 
directly unknown and are calculated from interpolations or
approximations.


\subsection{Hermite elements} % (fold)
\label{ssub:hermite_elements}
1D Hermite elements are polynomials ${\rm H}_{j}\left(\xi_i\right)$,
for which
\begin{equation}
\begin{array}{*{20}{c}}
  {{\rm H} _{j}^{\left( k \right)}\left( 0 \right) = {\delta _{jk}}} \\ 
  {{\rm H} _{j}^{\left( k \right)}\left( { \pm 1} \right) = 0} 
\end{array}
\end{equation}
where $\xi_i=\left( x - x_i \right) / h_i$, $- 1 \leqslant \xi  \leqslant 1$ is a normalized coordinate,
the superscript $^{\left( k \right)}$ denotes the $k^{\rm th}$
derivative and $\delta$ is the Kronecker symbol.
As such, the coefficients $F_{ij}$ in the finite element representation of
a function $f$,
\begin{equation}
f = \sum {{F_{ij}}{{\rm H} _{ij}}} 
\end{equation}
are directly the values of the $j^{\rm th}$
derivatives of $f$ at a mesh point $i$.
(Two indices are used for being more explicit; a renumbering
to a single index is straightforward.)

Cubic Hermite elements can be expressed as
\begin{equation}
{{\rm H}_0}\left( \xi  \right) = {\left( {\left| \xi  \right| - 1} \right)^2}\left( {2\left| \xi  \right| + 1} \right)
\end{equation}
% MathType!MTEF!2!1!+-
% feaagKart1ev2aaatCvAUfKttLearuqr1ngBPrgarmqr1ngBPrgitL
% xBI9gBamXvP5wqSXMqHnxAJn0BKvguHDwzZbqegm0B1jxALjhiov2D
% aeHbuLwBLnhiov2DGi1BTfMBaebbfv3ySLgzGueE0jxyaibaiKc9yr
% Vq0xXdbba91rFfpec8Eeeu0xXdbba9frFj0-OqFfea0dXdd9vqaq-J
% frVkFHe9pgea0dXdar-Jb9hs0dXdbPYxe9vr0-vr0-vqpWqaaeaabi
% GaciaacaqabeaadaabauaaaOqaaiabfE5ainaaBaaaleaacqaIXaqm
% aeqaaOWaaeWaaeaacqaH+oaEaiaawIcacaGLPaaacqGH9aqpcqaH+o
% aEcqWGObaAdaWgaaWcbaGaemyAaKgabeaakmaabmaabaWaaqWaaeaa
% cqaH+oaEaiaawEa7caGLiWoacqGHsislcqaIXaqmaiaawIcacaGLPa
% aadaahaaWcbeqaaiabikdaYaaaaaa!560E!
\begin{equation}
{{\rm H}_1}\left( \xi  \right) = \xi {h_i}{\left( {\left| \xi  \right| - 1} \right)^2}
\end{equation}
(Note that the scaling factor $h_i$ must be properly defined for
${\rm H}_1$.)

% subsection hermite_elements (end)


\subsection{Implications for CDE + G-S}

Advantages:
\begin{enumerate}
	\item Boundary conditions on derivatives are more natural (similar to Dirichlet).
	\item The solution, i.e. $\psi$, is $C^1$ (using cubic Hermite elements). 
	$\psi\left( x\right)$ and $\psi'\left( x\right)$ is directly known for any $x$.
	\item Some possibly noisy derivative terms from CDE can be eliminated.
	\item Perhaps, a finite element G-S solver (CEDRES++) might use $p$ and $F^2$ instead of $p'$ and $FF'$.
\end{enumerate}
Drawbacks:
\begin{enumerate}
	\item FEM is somewhat more complex to implement.
	\item For the integration, the equation coefficients should be given as functions. Spline interpolants are supposedly well suited.
\end{enumerate}

\subsection{B-spline elements}
Another possibility for the FE basis are B-splines.
In \cite{Felici2011}, cubic B-splines are used for the current diffusion equation.
Note that the cubic B-spline basis features a $C^2$ solution.
